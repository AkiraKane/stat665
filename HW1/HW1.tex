\documentclass[a4paper,10pt]{article}
\usepackage[utf8x]{inputenc}
\usepackage{amsfonts}
\usepackage{amsmath}

%opening
\title{Homework 1, STAT365}
\author{Girish Sastry}

\begin{document}

\maketitle

\section{Exercise 2.3}
Let $m$ be the median distance between the closest data point and the origin of the unit ball. Thus,
$$Pr[All N points distance \geq m) = \frac{1}{2}$$

Let $x_i$ be a single point and $d_i$ be its distance to the origin. Then,
$$Pr[d_i \geq m] = 1 - Pr[d_i < r) = 1 - \frac{C\pi m^p}{C\pi 1^p} = 1 - m^p$$

Thus,
  $$Pr[All N points distance \geq m] = \frac{1}{2} = \prod_{i=1}^N Pr[d_i > m] = (1 - m^p)^N.$$
  
Solving for m, we get
$$ m = \left(1 - \frac{1}{2}^\frac{1}{N}\right)^\frac{1}{p}$$

\section{Exercise 2.8}

In this exercise we compare classification of zipcode data using linear regression and k-nearest neighbors. 
To calculate the error for linear regression, we sum the difference of squares between $\vec{y_{test}}$ and $\vec{\hat{y}}$ for the test data
and the difference of squares between $\vec{y_{train}}$ and $\vec{\hat{y}}$ for the training data, and divide each by
the length of the vector to obtain the error. Note that this is equivalent to using R's inbuilt ``mean'' function. We proceed
by a similar error calculation for the k-nearest neighbors algorithm.

Results: \\

\begin{tabul``ar}{l*{6}{c}r}
 Method		& Test Error & Training Error \\
 \hline
 Linear Regression & $3.85\%$ & $0.58\%$ 			\\
 1-NN	& $2.47\%$ & $0.00\%$ \\
 3-NN 	& $3.02\%$ & $0.43\%$ \\
 5-NN	& $3.02\%$ & $0.58\%$ \\
 7-NN 	& $3.02\%$ & $0.58\%$ \\
 15-NN	& $3.85\%$ & $0.93\%$ \\
\end{tabular}

With these results, we see that linear regression performs a little worse than k-NN for small values of $k$. Both the 1-NN and
3-NN methods give better errors than linear regression. The 5-NN and 7-NN methods have better test error results, and the 15-NN
gives the worst results. The large number of predictors may help explain these results. And, as discussed in lecture and in the
text, we would expect lower values of $k$ to perform better. However, overall, the model performed well.


\section{Exercise 3.3}

		    
\subsection{Part a}

Please refer to Part b, which this justification will be based on. With $a'\beta$, we have a linear combination of $\beta$. And, just as in part b,
in this case, the sum of the variances and covariances in the linear combination will give us the total variance. And since each $v_{i}$ and covariance in the linear combination
	     is less than its counterparts in the linear combination for $\hat{\theta}$ (as the variance-covariance matrix exceeds the other variance-covariance
	     matrix by a semidefinite matrix), then the $var(a'\beta) \leq var(a'\hat{\theta})$.
	     Thus, the proof for part b will hold for
part a.

\subsection{Part b}
As in the text, let $\hat{\theta} = CY$ be a different linear estimator of $\hat{\beta}$ and let $C = (X'X)^{-1}X' + D$ where $D$ is a $k$ x $n$ nonzero matrix.
Then,
\begin{align} V(\hat{\theta}) &= V(CY) = CV(Y)C' = \sigma^2CC' \\
  &= \sigma^2((X'X)^{-1} + D)(X(X'X)^{-1} + D') \\
    &= \sigma^2((X'X)^{-1}X'X(X'X)^{-1} + (X'X)^{-1}X'D' + DX(X'X)^{-1} + DD') \\
    &= \sigma^2(X'X)^{-1} + \sigma^2(X'X)^{-1}(DX)' + \sigma^2DX(X'X)^{-1} + \sigma^2DD' \\
    &= \sigma^2(X'X)^{-1} + \sigma^2DD' \\
    &= V(\hat{\beta}) + \sigma^2DD'
\end{align}

Because $DD'$ is positive semidefinite, $$V(\hat{\theta}) \succeq V(\hat{\beta})$$


\section{Appendix}

This section contains the R code for Exercise 2.8.

\end{document}
